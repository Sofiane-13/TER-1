%%%%%%%%%%%%%%%%%%%%%%%%%%%%%%%%%%%%%%%%%
% Arsclassica Article
% LaTeX Template
% Version 1.1 (1/8/17)
%
% This template has been downloaded from:
% http://www.LaTeXTemplates.com
%
% Original author:
% Lorenzo Pantieri (http://www.lorenzopantieri.net) with extensive modifications by:
% Vel (vel@latextemplates.com)
%
% License:
% CC BY-NC-SA 3.0 (http://creativecommons.org/licenses/by-nc-sa/3.0/)
%
%%%%%%%%%%%%%%%%%%%%%%%%%%%%%%%%%%%%%%%%%

%----------------------------------------------------------------------------------------
%	PACKAGES AND OTHER DOCUMENT CONFIGURATIONS
%----------------------------------------------------------------------------------------

\documentclass[
10pt, % Main document font size
a4paper, % Paper type, use 'letterpaper' for US Letter paper
oneside, % One page layout (no page indentation)
%twoside, % Two page layout (page indentation for binding and different headers)
headinclude,footinclude, % Extra spacing for the header and footer
BCOR5mm, % Binding correction
]{scrartcl}

\input{structure.tex} % Include the structure.tex file which specified the document structure and layout

\hyphenation{Fortran hy-phen-ation} % Specify custom hyphenation points in words with dashes where you would like hyphenation to occur, or alternatively, don't put any dashes in a word to stop hyphenation altogether

%----------------------------------------------------------------------------------------
%	TITLE AND AUTHOR(S)
%----------------------------------------------------------------------------------------

\title{\normalfont\spacedallcaps{Faut-il tester le code ou les modèles de composants logiciels ?}} % The article title

\date{} % An optional date to appear under the author(s)

%----------------------------------------------------------------------------------------

\newcommand{\HRule}{\rule{\linewidth}{0.5mm}}

\addto\captionsfrench{% Replace "english" with the language you use
  \renewcommand{\contentsname}
    {Sommaire}
}

\begin{document}

\begin{titlepage}
  \begin{sffamily}
  \begin{center}

    % Upper part of the page. The '~' is needed because \\
    % only works if a paragraph has started.
    \includegraphics[scale=0.2]{Figures/univnantes.png}~\\[1.5cm]

    \textsc{\LARGE Université de Nantes}\\[0.5cm]
    \textsc{\Large Master Informatique}\\[2cm]
    \textsc{\Large Rapport de stage TER}\\[1.5cm]

    % Title
    \HRule \\[0.4cm]
    { \huge \bfseries Faut-il tester le code ou les modèles de composants logiciels ?\\[0.4cm] }

    \HRule \\[3cm]
    %\includegraphics[scale=0.2]{Figures/Dolor.JPG}
    %\\[2cm]

    % Author and supervisor
    \begin{minipage}{\textwidth}
      \begin{flushleft} \large
        \textsc{\Large Réalisé par : } \\
        Daniel \textsc{Ahmed}\\
        Yamen \textsc{Alnajm}\\
        Soufyane \textsc{Belhadj Kacem}\\
        Promo 2018\\
      \end{flushleft}
    \end{minipage}
    \begin{minipage}{\textwidth}
      \begin{flushright} \large
        \emph{Tuteur :} \\
         M. \textsc{Gilles Ardourel}\\
         M. \textsc{Pascal Andre}\\
         M. \textsc{Jean-Marie Mottu}\\
        \emph{Chef d'équipe : } \\
         M. \textsc{Gile}
      \end{flushright}
    \end{minipage}

    \vfill

    % Bottom of the page
    %{\large 1\ier{} Juillet 2013 — 30 Août 2013}

  \end{center}
  \end{sffamily}
\end{titlepage}

%----------------------------------------------------------------------------------------
%	HEADERS
%----------------------------------------------------------------------------------------

\renewcommand{\sectionmark}[1]{\markright{\spacedlowsmallcaps{#1}}} % The header for all pages (oneside) or for even pages (twoside)
\renewcommand{\subsectionmark}[1]{\markright{\thesubsection~#1}} % Uncomment when using the twoside option - this modifies the header on odd pages
\lehead{\mbox{\llap{\small\thepage\kern1em\color{halfgray} \vline}\color{halfgray}\hspace{0.5em}\rightmark\hfil}} % The header style

\pagestyle{scrheadings} % Enable the headers specified in this block



%----------------------------------------------------------------------------------------
%	TABLE OF CONTENTS & LISTS OF FIGURES AND TABLES
%----------------------------------------------------------------------------------------

\maketitle % Print the title/author/date block


\setcounter{tocdepth}{2} % Set the depth of the table of contents to show sections and subsections only

\tableofcontents % Print the table of contents
\newpage
\listoffigures % Print the list of figures
\newpage
\listoftables % Print the list of tables

\newpage
\section*{Remerciements} 
Nous tenons  à remercier cordialement toutes les personnes qui nous ont permis de réaliser ce travail dans les meilleures conditions.\\ 

Un remerciement particulier à Mr. Gilles ARDOUREL et Pascal ANDRÉ, nos enseignants responsables, pour nous avoir encadrés tout au long de ce travail. Merci pour leurs conseils, leurs disponibilités et leurs implications. \\

Nous remercions les membres du jury qui ont bien voulu examiner et évaluer ce rapport.

%----------------------------------------------------------------------------------------
%	ABSTRACT
%----------------------------------------------------------------------------------------
\newpage
\section*{Résumé}

Ce sujet de rapport vise à étudier et appliquer l'approche d'ingénierie dirigée par les modèles pour le développement d’une application qui automatise les tests  logiciels au niveau du modèle.
Le Budget IT consacrée aux tests et à l’assurance qualité augmente de 9 points d’une année sur l’autre, pour atteindre 35 \% de la dépense IT selon Capgemini. Contre seulement 26 \% en 2014 ! Et 18 \% en 2012.\\

La SSII prévoit qu’en 2018, la part des tests et de la qualité passera à 40 \% du total. Ce budget colossal est divisé en parts égales entre maintenance d’applications existantes et nouveaux développement. Dans 35 \% des entreprises, cette inflation est considérée comme un problème.\\

Cette problématique pousse les entreprises à adopter une nouvelle stratégie et de miser sur la recherche d’un mécanisme d’automatisation des tests logiciels.\\

Cela nous ramène à penser à une solution pour automatiser les tests logiciel au niveau modèle, en se basant sur l’approche MDA, le but est d’offrir une interface qui donne la possibilité à l’utilisateur de choisir les composants ainsi que les services qui veulent tester, et notre application se charge du reste.\\

Toute la partie de développement de ce projet fut mis en place de manière itérative, basée sur la méthodologie Agile.

\section*{Abstract} % This section will not appear in the table of contents due to the star (\section*)

\lipsum[1] % Dummy text


\section*{Mots-clés : }

%----------------------------------------------------------------------------------------
%	AUTHOR AFFILIATIONS
%----------------------------------------------------------------------------------------

%\let\thefootnote\relax\footnotetext{* \textit{Department of Biology, University of Examples, London, United Kingdom}}

%\let\thefootnote\relax\footnotetext{\textsuperscript{1} \textit{Department of Chemistry, University of Examples, London, United Kingdom}}

%----------------------------------------------------------------------------------------

\newpage % Start the article content on the second page, remove this if you have a longer abstract that goes onto the second page

%----------------------------------------------------------------------------------------
%	INTRODUCTION
%----------------------------------------------------------------------------------------

\section{Introduction}

L’architecture pilotée par les modèles (MDA) permet la modélisation de système à composant et des services. Elle permet de s’abstraire de la technologie qui sera utilisée lors du passage du modèle au système. Cette même philosophie pourrait être appliquée aux tests logiciels. On parle ici de crée un modèle pour abstraire l’écriture de test logiciel. La modélisation des tests permettrait de repérer les erreurs de conception à un stade précoce du processus de conception.
 
%----------------------------------------------------------------------------------------
%	METHODS
%----------------------------------------------------------------------------------------

\newpage 
\section{L'existant}

\subsection{L'outil Kmelia}

\lipsum[5] % Dummy text

\subsection{L'architecture du système}

\lipsum[5] % Dummy text

\subsection{L'outil CostoTest}

\lipsum[5] % Dummy text


\newpage 
\section{Nouvelle approche}
\lipsum[5] % Dummy text

\subsection{L'intention de test}
\lipsum[5] % Dummy text

\subsection{L'assistance au harnais de test}
\lipsum[5] % Dummy text


%----------------------------------------------------------------------------------------
%	RESULTS AND DISCUSSION
%----------------------------------------------------------------------------------------

\newpage 
\section{Résultats et Discutions}



\lipsum[10] % Dummy text


%----------------------------------------------------------------------------------------
%	BIBLIOGRAPHY
%----------------------------------------------------------------------------------------

\renewcommand{\refname}{\spacedlowsmallcaps{References}} % For modifying the bibliography heading
\newpage 
\bibliographystyle{dinat}

\bibliography{biblio.bib} % The file containing the bibliography

%----------------------------------------------------------------------------------------

\end{document}